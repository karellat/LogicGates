% !TEX encoding = UTF-8 Unicode
\documentclass[12pt, oneside]{article}  
\usepackage[T1]{fontenc}
\usepackage[utf8]{inputenc}
\usepackage[czech]{babel}					
\usepackage{amsmath}
\usepackage{amssymb}
\usepackage{graphics}
\usepackage{listings} 
\usepackage{graphicx}
\usepackage{forest}
\usepackage{fullpage}
\usepackage{cancel}
\newcommand\tab[1][1cm]{\hspace*{#1}}
\newcommand*{\QEDB}{\hfill\ensuremath{\square}}
\title{\vspace{-12ex}Logic Gates \\ C++ zápočtový }
\author{\vspace{-10ex}Tomáš Karella}
\date{\today}
\begin{document}
\maketitle
\section*{Téma:} Cílem zápočtového programu je program implementující logické sítě a rozhraní, které umí načíst konstrukt hradla ze souboru.
\section*{Kompilace a spouštění:} Pro spuštění na Linux distribucích je nutný překladač g++-6 a GNU Make.  \\
\begin{lstlisting} [language=bash]
make
\end{lstlisting} 
Zkompiluje zdrojové soubory a spustí interaktivní režim. \\
\\ \\
Pro spuštění na Windows distribucích je program dostupný pouze jako Visual Studio projekt(VS 2015 a vyšší) \\ 
\section*{Uživatelská dokumentace:}
Po spuštění bez parametrů se otevře interaktivní režim, který vás vyzve k zadání cesty konstručkního souboru. Po jeho úspěšném zkonstruování se přepne do  režimu vkládání vstupu, kdy pro daný vstup spočítá a vrátí výstup. Dále poskytuje možnost zkonstruovat hradlo pro další použití(,klíčové slovo: c).  Po úspěšné konstrukci se opět dostane do režimu načítání souborů. Nyní už může používat jméno prvního konstruovaného hradla jako typ. \\
\subsection*{Klíčová slova:} 
\textbf{exit,e} - slouží k ukončení aplikace \\
\textbf{const, c} - konstruuje zadané hradlo, (jen v režimu, kdy je načtený konstrukční soubor) \\
\textbf{h,help,man} - zobrazí klíčová slova \\
\subsection*{Konstruční soubor - formát:}
Modelový soubor lze nalézt "examples/model.txt".  Ve zmíněné složce je i celá řada příkladů k vyzkoušení programu. \\
Soubor se skládá ze dvou hlavních částí. Pojmenování hradel, kde deklarujete jméno hradla(noCASE sensitive a smí obsahovat pouze číslice a písmena) k jménu typ hradla. Část druhá, kde se řeší jejich vzájemné propojení. Jednotlivé tagy jsou odděleny tabulátorem. \\
\\ \\
\#GATE\tab MYNAME (1) \\
NameOfGate \tab Type (2)\\
NameOfGate	\tab Type \\
\#CONNECT (3) \\
NameOfGate[OutputPinID]\tab	-> \tab	NameOfGate[InputPinID] (4) \\
NameOfGate[OutputPinID]	 \tab -> \tab	NameOfGate[InputPinID] \\
\# (5) 
\\ \\ 
\begin{enumerate} 
\item kontrolní tag pro pojmenovácí část souboru a jméno vašeho hradla (oddělené tabulátorem)
\item jméno hradla(pouze písmena a číslice) dále typ(predefinovaný či už zkonstruovaný) vzájemně odděleny tabulátorem. 
\item  kontrolní tag pro začátek propojovací část souboru
\item  jméno hradla a v hranatých závorkách číslo výstupního pinu dále "->" oddělená z obou stran tabulátorem jméno hradla s číslem vstupního pinu. 
\end{enumerate} 
Pro konstrukci musí být připojeny u hradel všechny vstupní i výstupní piny, konstrukt musí obsahovat alespoň jedno hradlo vstupní a alespoň jedno výstupní. 
\subsection*{Konstručkní soubor - příklady} 

\end{document}
